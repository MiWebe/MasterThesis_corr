In many, especially arid and semi-arid regions groundwater is a major source for drinking water, industrial use and agriculture. 
Due to intensification of agricultural activities and population growth, stresses on groundwater resources increased in recent decades (\cite{Choukr.2017}, \cite{ElRawy.2022}). 
This has led to groundwater over-exploitation, inducing a decrease of groundwater tables with various consequences \parencite{Hssaisoune.2017}. 
One example is that former productive wells dried out, which raised the necessity of drilling new and deeper wells that are more costly \parencite{English.1990}. 

Another effect can be saltwater intrusion in coastal aquifers \parencite{Fetter.2001} - there, a natural balance exists between freshwater from inland and more dense saltwater from the ocean \parencite{Mays.2005}. 
If the groundwater table inland decreases due to groundwater overexploitation, this balance is disrupted \parencite{Fetter.2001}. 
In absence of counterpressure from the land side, saltwater pushes into the aquifer. 
Wells in proximity to the coast, that were previously pumping freshwater are contaminated, as they can succeedingly only provide saltwater \parencite{Fetter.2001}. 
This saltwater however is unfit for common agricultural use \parencite{Taiz.2015} and other purposes.

Countermeasures to mitigate groundwater depletion can be the reduction of water usage. 
Therefore, modernisation of irrigation techniques to more water-efficient methods, implementation of deficit irrigation, reduction of tap water usage, increases of water cycle efficiencies in urban areas or the cultivation of different plant species with lower water demand are examples (\cite{English.1990}, \cite{Schuetze.2012}). 
Second, the quantity of available freshwater resources can be increased, e.g. through operation of desalinisation plants in coastal areas or the import of water from other regions. 
Third, the microclimate conditions can be manipulated favourably through changes in land usage.

However, any of these countermeasures require detailed assessments of their particular plausibility and economical efficiency. 
Therefore, robust estimations of future water availability and the single measures', site-specific impact on water balances are required \parencite{ElRawy.2022}. 
Therewith, possible scenarios can be developed. A valuable tool for precise estimations of impacts of possible measures are numerical models of the hydrological system \parencite{ElRawy.2022}.

As countries with semi-arid climate, Morocco, Algeria and Tunisia founded the Regional Cooperation for a Sustainable Management of the Water Resources in the Mahgreb (CREM) to address regional issues water scarcity \parencite{OSS.2023}. 
One goal of this project is the implementation of a regional strategy for sustainable water resource management. 
To achieve this, CREM represents a technical cooperation with the German Office for International Cooperation (GIZ) and the Federal Institute for Geosciences and Natural Resources (BGR). 
The BGR module focuses on the data-based management of water resources, including among others numerical models of aquifers as tools for decision making. 

Within this study - as a part of the BGR module - a corresponding steady state groundwater flow model is extended to a transient groundwater flow model for the agriculturally important Chtouka aquifer in Morocco. 
The Chtouka aquifer underlies the Chtouka plain, a major agricultural area located within the Souss-Massa River Basin in south-western Morocco. 
Freshwater in the Chtouka plain is solely supplied by the Youssef Ben Tachfine (YBT) water reservoir and the Chtouka aquifer. 
In Chapter \ref{Chap-SouMas} this area is described in detail. 
In a previous study by \textcite{Horn.2021}, a steady state model was calibrated for the hydrological year 1968/69. 
It is implemented in the Groundwater Modelling System (GMS). 
In this study, it is expanded to a transient constant-density groundwater flow model. 
The conceptual model is presented and the adjustments made are described in Chapter \ref{Chap-ConcMod}. 
The implementation into GMS is then described in Chapter \ref{Chap-ImplMod}. 
For the sensitivity analysis and the calibration of the model, the common error measures root mean squared error ($RMSE$) and mean absolute error ($MAE$) are deemed to be inaccurate. 
Therefore, three custom error measures are identified and partially derived from considerations of the correlations between observations and simulation results (Chapter \ref{Chap-ImplMod}). 
These are coefficients describing the over- or underestimation of initial groundwater levels ($\Delta S_0$), long-term trends ($p_1$) and deviations from a linear correlation ($R^2$). 
A sensitivity analysis of the implemented model is executed and the model is calibrated using the three error measures. 
The performance of the measures is then discussed and compared to the $RMSE$. 
Finally, the results of the model are presented.