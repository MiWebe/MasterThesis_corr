\thispagestyle{plain}
\begin{center}
    \LARGE
    \textbf{Abstract}\\
    \vspace{1cm}
    \hspace{0pt}
    \large
    Michael Weber \\
    \vspace{0.5cm}
    \small
    Student at Leibniz Universität Hannover\\
    weber@stud.uni-hannover.de

    \vspace{0.7cm}
    \pagenumbering{gobble}
\end{center}

%\chapter*{Abstract}
\normalsize
The Chtouka plain in South-Western Morocco represents an important contributor to federal agricultural production. As it is located in a region of semi-arid climate, extensive irrigation is required to sustain the cultivated plant species. Due to lack of sufficient surface water sources, water demand is met by groundwater extraction from the Chtouka aquifer. Since the midst of the $20^\textrm{th}$ century groundwater overexploitation developed which led to significant drawdowns of the local groundwater table of $20 \, \textrm{m}$ and more in recent years. This drawdown results in intrusion of saltwater from the adjoining Atlantic Ocean into coastal areas and further depletes available freshwater resources. On administrative level, in the last two decades a number of measures have been taken to counteract this development. However, the assessment of these and future management decisions heavily relies on information about the state and response of the aquifer. For this, groundwater models have proven a valuable instrument to predict impacts of measures and future developments.

Within a preceding work by \textcite{Horn.2021}, a geological model of the aquifer was derived and a steady state variable density groundwater flow model for the year 1968 was implemented into the groundwater modelling software GMS. In this study, the underlying groundwater flow model for constant density is extended to a transient model. Furthermore, additional adjustments are done to the underlying conceptual model to achieve a more accurate representation of the real system.   Succeedingly, the model is calibrated for transient state in respect to a set of eleven model parameters. For this process however, the commonly used error measures $RMSE$ and $MAE$ are considered as theoretically unfit for assessment of time series. Therefore a set of three custom error measures is derived, which characterise the initial offset of simulated hydraulic heads, their long-term trends and the accuracy of the underlying assumptions in comparison to the observational data. With these measures, a sensitivity analysis and the subsequent calibration are executed. A final set of parameter values is obtained that fits the simulation values best to the observations. Finally, the custom error measures are compared to the $RMSE$. It is found that in this particular case both are equivalent. Finally it is concluded that even though no benefit on the result in the present application can be observed, the derived error measures however can improve calibrations in other cases when parameters show a higher level of interaction.\\[1.0cm]
\vspace{0.5cm}
\textbf{Keywords} \hspace{0.3cm} groundwater modelling $\cdot$ transient state $\cdot$ model calibration