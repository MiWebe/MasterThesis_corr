\thispagestyle{plain}
\begin{center}
    \LARGE
    \textbf{Résumé}\\
    \vspace{1cm}
    \hspace{0pt}
    \large
    Michael Weber \\
    \vspace{0.5cm}
    \small
    Étudiant à l'Université Leibniz de Hanovre\\
    weber@stud.uni-hannover.de

    \vspace{0.7cm}
    \pagenumbering{gobble}
\end{center}

%\chapter*{Abstract}
\normalsize

La plaine de Chtouka, dans le sud-ouest du Maroc, contribue de manière importante à la production agricole nationale. Comme elle est située dans une région au climat semi-aride, une irrigation extensive est nécessaire pour maintenir les cultures. En raison de l'absence de ressources en eau de surface suffisantes, la demande en eau est satisfaite par l'extraction des eaux souterraines de l'aquifère de Chtouka. Depuis le milieu du 20e siècle, la surexploitation des eaux souterraines a conduit localement à des baisses significatives de la nappe phréatique de 20 m et plus ces dernières années. Ce rabattement entraîne l'intrusion d'eau salée de l'océan Atlantique dans les zones côtières et un appauvrissement supplémentaires des ressources en eau douce disponibles. Au niveau administratif, un certain nombre de mesures ont été prises au cours des deux dernières décennies pour contrer cette évolution. Cependant, l'évaluation de ces mesures et les décisions de gestion futures dépendent fortement des informations sur l'état et la réponse de l'aquifère. Pour cela, les modèles d'écoulement d'eau souterraine se sont avérés être un instrument précieux pour prédire les impacts des mesures et des développements futurs.

Dans un travail précédent de \textcite{Horn.2021}, un modèle géologique de l'aquifère a été développé et un modèle d'écoulement des eaux souterraines à densité variable en régime permanent pour l'année 1968 a été implémenté dans le logiciel GMS (Groundwater Modeling Software). Dans cette étude, un modèle d'écoulement transitoire à densité constante pour 1968-2020 est développé sur la base du modèle de \textcite{Horn.2021}. En outre, des ajustements supplémentaires sont effectués sur le modèle conceptuel sous-jacent afin d'obtenir une représentation plus précise du système réel. Ensuite, le modèle est calibré en régime transitoire en ajustant onze paramètres. Cependant, les mesures d'erreur communément utilisées RMSE et MAE sont considérées comme théoriquement inadaptées à l'évaluation des séries temporelles. Par conséquent, un ensemble de trois mesures d'erreur spécifiques est dérivé, qui caractérise: le décalage initial des hauteurs hydrauliques simulées, leurs tendances à long terme et l'exactitude des hypothèses sous-jacentes par rapport aux données d'observation. Avec ces mesures, une analyse de sensibilité et la calibration en résultant sont effectuées. Un ensemble final de valeurs de paramètres est obtenu qui ajuste au mieux les valeurs de simulation aux observations. Enfin, les mesures d'erreur spécifiques sont comparées à la RMSE. Il s'avère que dans ce cas particulier, les deux sont équivalentes. Enfin, il est conclu que même si aucun avantage sur le résultat dans la présente application ne peut être observé, les mesures d'erreur dérivées pourraient cependant améliorer les calibrations dans d'autres cas où les paramètres montrent un niveau d'interaction plus élevé.\\[1.0cm]
\vspace{0.5cm}
\textbf{Mots clés} \hspace{0.3cm} modélisation des eaux souterraines $\cdot$ régime transitoire $\cdot$ calibration de modèle