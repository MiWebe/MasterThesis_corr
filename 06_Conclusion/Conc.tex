In this study, the steady state model by \textcite{Horn.2021} could successfully be expanded to a transient constant-density flow model. 
On this, future work can further advance by generalising it to a variable density model. 
Therewith, possible scenarios of management decisions and environmental influences can be examined on their impacts on the system, to facilitate sustainable development of the water resources in the region.

Although there are no direct indications that through the derived parameters a significantly different result could be obtained than under application of the $RMSE$, the examined error measures may show greater impacts in calibration processes of other models where a larger interaction between different parameters exists and single parameters may influence several different characteristics of a model.

As the final results show, there might be an inaccuracy of $\Delta S_0$. 
This however has not yet been examined further in this study. 
For future applications it is therefore of interest, if the interpreted meaning of $\Delta S_0$ holds against reality. 
To test this with the here examined data, time series dating back to the initial date 1968 are valuable. 
With these a testing might be possible. 
Therefore a randomised process can be used, which generates - not necessarily disjunct - subsets of consecutive years time series with varying lengths. 
The estimates of $\Delta S_0$ can for these then be compared to the actual initial offsets. 
Through repeated executions of the randomised process, an average error of $\Delta S_0$ can be determined. 
However, it is to be noted with this method that only a small number of specific cases is considered, which could be inapt for the testing.